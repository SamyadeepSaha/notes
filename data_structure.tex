\section{Data Structure}

\subsection{Array}
Array is a collection of items of the same variable type that are stored
at contiguous memory location. It is one of the most popular and simple
data structures used in programming.

\textbf{Basic terminologies of Array}

\begin{itemize}
    \item Array Index: In an array, elements are identified by their
        indexes. Array index starts from 0.
    \item Array Element: Elements are items stored in an array and can
        be accessed by their index.
    \item Array Length: The length of an array is determined by the
        number of elements it can contain.
\end{itemize}

\textbf{Memory representation of Array:}

In an array, all the elements are stored in contiguous memory locations.
So, if we intitialize an array, of elements.

\textbf{Purpose of arrays:}

Assume there is a class of five students and if we have to keep records
of their marks and examination then, we can do this by declaring five
variables individual and keeping track of records but what if the number
of students becomes very large, it would be challenging to manipulate
and maintain the data.

What it means is that, we can use normal variables (v1, v2, v3,...)
when we have a small number of objects. But if we want to store a large
number of instances, it becomes difficult to manage them with normal
variables. The idea of an array is to represent many instances in one
variable.

\textbf{Types of Arrays}

\begin{itemize}
    \item On the basis of Size
    \item On the basis of Dimensions
\end{itemize}

\textbf{Types of Arrays on the basis of Size:}

\begin{enumerate}
    \item Fixed Size Arrays:
        We Cannot alter or update the size of this array. Here only a
        fixed size (i,e. the size that is mentioned in square bracketes
        []) of memory will be allocated for storage. In case, we don't
        know the size of the array then if we declare a larger size and
        store a lesser number of elements will result in a wastage of
        memory or we declare a lesser size than the number of elements
        then we won't get enough memory to store all the elements. In
        such cases, static memory allocation is not preferred.
    \item Dynamic Sized Arrays:
        The size of the array changes as per user requirements during
        execution of cod so the coders do not have to warry about sizes.
        They can add and removed the elements as per the need. The
        memory is mostly dynamically allocated and de-allocated in these
        arrays.
\end{enumerate}

\textbf{Types of Arrays on the basis of Dimensions:}

\begin{enumerate}
    \item One-dimensional Array: You can imagine a 1d array as a row,
        where elements are stored one after another.
    \item Multi-dimensional Array: A multi -dimensional array is an
        array with more than one dimension. We can use multidimensional
        array to store complex data in the form of tables, etc. We can
        have 2D arrays, 3D arrays, 4D arrays and so on.
        \begin{enumerate}
            \item Two-Dimensional Array(Matrix): 2D Mulitdimensional
                arrays can be considered as an array of arrays or as a
                matrix consisting of rows and columns.
            \item Three-Dimensional Array: A 3D Mulitdimensional array
                contains three dimensions, so it can be considered an
                array of two-dimensional arrays.
        \end{enumerate}
\end{enumerate}

\textbf{Application of Array Data Structure:}

Array mainly have advantages like random access and cache friendliness
over other data structures that make them useful.
Below are some applications of arrays.
\begin{enumerate}
    \item Storing and accessing data: Arrays store elements in a
        specific order and allow constant-time O(1) access to any
        element.
    \item Searching: If data in array is storted, we can search an item
        in O(log n) time. We can also find floor(), ceiling(), kth
        smallest, kth largest, etc efficiently.
    \item Matrices: Two-dimensional arrays are used for matrices in
        computations like graph algorithms and image processing.
    \item Implementing other data structures: Arrays are used as the
        underlying data structure for implementing stacks and queues.
    \item Dynamic programming: Dynamic programming algorithms often use
        arrays to store intermediate results of subproblems in order to
        solve a larger problem.
    \item Data Buffers: Arrays serve as data buffers and queues,
        temporarily storing incoming data like network packets, file
        streams, and database results before processing.
\end{enumerate}


\subsection{Heap Data Structure}
A Heap is a complete binary tree data structure that satisfies the heap
property: for every node, the value of its children is greater than or
equal to its own value. Heaps are usually sused to implement priority
queues, where tha smallest (or largest) element is always at the root of
the tree.

\subsubsection{Binary Heap}
A Binary Heap is a complete binary tree which is used to store data
efficiently to get the max or min element based on its type. A Binary
Heap is either Min Heap or Max Heap. In a Min Binary Heap, the key at
the root must be minimum among all keys present in Binary Heap. The same
property must be recursively true for all nodes in Binary Tree. Max
Binary Heap is similar to Min Heap.

\textbf{How is Binary Heap represented?}
A Binary Heap is a Complete Binary Tree. A binary heap is typically
represented as an array.
\begin{enumerate}
    \item The root element will be arr[0].
    \item The below table shows indices of other nodes for the ith node,
        i.e., arr[i]:
        \begin{center}
            \begin{tabular}{|c|c|}
                \hline
                arr[(i-1)/2] & Returns the parent node \\
                \hline
                arr[(2*i)+1] & Returns the left child node \\
                \hline
                arr[(2*i)+2] & Return the right child node \\
                \hline
            \end{tabular}
        \end{center}
\end{enumerate}
