\section{Mathematic Logic}

\subsection{Propositional and Predicate Logic}

\subsection{Normal and Principle Forms}

\subsubsection{Disjunctive Normal Forms(DNF)}
A formula which is equivalent to a given formula and which consists of a
sum of elementary products is called a disjunctive normal form of given
formula. \textbf{Example} $ (p \land \neg q) \lor (q \land r) \lor (\neg
p \land q \land \neg r) $. The DNF of formula is not unique.

\subsubsection{Conjunctive Normal Form(CNF)}
A formula which is equivalent to a given formula and which consists of a
product of elementary sums is called a conjunctive normal form of given
formula. \textbf{Example} $ (p \lor \neg q) \land (q \lor r) \land (\neg
p \lor q \lor \neg r) $. The CNF of formula is not unique. If every
elementary sum in CNF is tautology, then given formula is also
tautology.

\subsubsection{Principle Disjunctive Normal Form (PDNF)}
An equivalent formula consisting of disjunctions of minterms only is
called the principle disjunctive normal form of the formula. It is also
known as \textbf{sum of products} canonical form. \textbf{Example} $ (p
\land \neg q \land \neg r) \lor (p \land \neg q \land r) \lor ( \neg p
\land \neg q \land \neg r) $. The minterm consists of conjuctions in
which each statement variable or its negation, but not both, appears
only once. The minterms are written down by including the variable if
its truth value is T and its negation if its truth value is F.

\subsubsection{Principle Conjunctive Normal Form (PCNF)}
An equivaletn formula consisting of conjunctions of maxterms only is
called the principle conjunctive normal form of the formula. It is also
known as \textbf{product of sums} canonical form. \textbf{Example} $ (p
\lor \neg q \lor \neg r) \land (p \lor \neg q \lor r) \land (\neg p \lor
\neg q \lor \neg r) $. The maxterm consists of disjunctions in which
each variable or its negation, but not both, appears only once. The dual
of a minterm is called maxterm. Each of the maxterm has the truth value
F for exactly one combination of the truth values of the variables. The
maxterms are written down by including the variable if its truth value
is F and its negation if its truth value is T.
