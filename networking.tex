\section{Networking}

\subsection{Classes of IPv4 addresses}
In the IPv4 IP address space, there are five classes A, B, C, D and E.
Each class has a specific range of IP addresses. Primarily, class A, B
and C are used by the majority of devicese on the Internet. Class D and
E are for special uses.

\subsubsection{Class A}
Class A addresses are for networks with large number of total hosts.
Class A allows for 126 networks by using the first octet for the network
ID. The first bit in this octet, is always zero. The remaining seven
bits in this octet complete the network ID. The 24 bits in the remaining
three octets represent the hosts ID and allows for approximately 17
million hosts per network. Class A network number values begin at 1 and
end at 127.
\begin{itemize}
\item Public IP range: 1.0.0.0 to 127.0.0.0 (First octet value range
from 1 to 127)
\item Private IP range: 10.0.0.0 to 10.255.255.255
\item Subnet mask: 255.0.0.0 (8 bits)
\item Number of networks: 126
\item Number of hosts per network: 16,777,214
\end{itemize}

\subsubsection{Class B}
Class B addresses are for medium to large sized networks. Class B allows
for 16,384 networks by using the first two octets for the network ID.
The first two bits in the first octet are always 1 0. The remaining six
bits, together with the second octet, complete the network ID. The 16
bits in the third and fourth octet represent host ID and allows for
approximately 65,000 hosts per network. Class B network number values
begin at 128 and end at 191.
\begin{itemize}
    \item Public IP range: 128.0.0.0 to 191.255.0.0, first octet value
        range from 128 to 191
    \item Private IP Range: 172.16.0.0 to 172.31.255.255
    \item Subnet Mask: 255.255.0.0(16 bits)
    \item Number of networks: 16,382
    \item Number of hosts per Network: 65,534
\end{itemize}

\subsubsection{Class C}
Class C addresses are used in small local area networks(LANs). Class C
allows for approximately 2 million networks by using the first three
octets for the network ID. In a class C IP address, the first three bits
of the first octet are always 1 1 0. And the remaining 21 bits of first
three octets complete the network ID. The last octet (8 bits) represent
the host ID and allows for 254 hosts per network. Class C network number
values begins at 192 and end at 223.
\begin{itemize}
    \item Public IP range: 192.0.0.0 to 223.255.255.0, first octet value
        range from 192 to 223
    \item Private IP range: 192.168.0.0 to 192.168.255.255
    \item Special IP range: 127.0.0.1 to 127.255.255.255
    \item Subnet mask: 255.255.255.0(24 bits)
    \item Number of networks: 2,097,150
    \item Number of hosts per network: 254
\end{itemize}

\subsubsection{Class D}
Class D IP addresses are not allocated to hosts and are used for
multicasting. Multicasting allows a single host to send a single stream
of data to thousands of hosts across the Internet at the same time. It
is often used for audio and video streaming, such as IP-based cable TV
networks. Another example is the delivery of real-time stock market data
from one source to many brokerage companies.
\begin{itemize}
    \item Range: 224.0.0.0 to 239.255.255.255, first octet value range
        from 224 to 239
    %\item Number of network: N/A
    %\item Number of hosts per network: Multicasting
\end{itemize}

\subsubsection{Class E}
Class E IP addresses are not allocated to hosts and are not available
for general use. Theses are reserved for research purposes.
\begin{itemize}
    \item Range: 240.0.0.0 to 255.255.255.255, first octet value range
        from 240 to 255
\end{itemize}

\subsubsection{Special IP Addresses}
IP range: 127.0.0.1 to 127.255.255.255 are network testing addresses
(also referred to as loop-back addresses). These are virtual IP address,
in that they cannot be assigned to a device. Specially, the IP
127.0.0.1 is often used to troubleshoot network connectivity issues
using the ping command. It tests a computer's TCP/IP network software
driver to ensure it is working properly.
