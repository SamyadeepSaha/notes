\section{Linear Programming Problem}

\subsection{Duality}
Every LPP called the primal is associated with another LPP called dual.
Either of the problems is primal with the other one as dual. The optimal
solution of either problem reveals the information about the optimal
solution of the other.

Let the primal problem be,
$$ Max Z_{x} = c_{1}x_{1} + c_{2}x_{2} + \ldots + c_{n}x_{n} $$

Subject to restrictions
$$ a_{11}x_{1} + a_{12}x_{2} + \ldots + a_{1n}x_{n} \leq b_{1} $$
$$ a_{21}x_{1} + a_{22}x_{2} + \ldots + a_{2n}x_{n} \leq b_{2} $$
$$ \vdots $$
$$ a_{m1}x_{1} + a_{m2}x_{2} + \ldots + a_{mn}x_{n} \leq b_{m} $$

and
$$ x_{1} \geq 0, x_{2} \geq 0, x_{n} \geq 0 $$

The corresponding dual is defined as
$$ Min Z_{w} = b_{1}w_{1} + b_{2}w_{2} + \ldots + b_{n}w_{n} $$

Subjebt to restribtions
$$ a_{11}w_{1} + a_{12}w_{2} + \ldots + a_{1n}w_{n} \leq c_{1} $$
$$ a_{21}w_{1} + a_{22}w_{2} + \ldots + a_{2n}w_{n} \leq c_{2} $$
$$ \vdots $$
$$ a_{m1}w_{1} + a_{m2}w_{2} + \ldots + a_{mn}w_{n} \leq c_{m} $$

and
$$ w_{1} \geq 0, w_{2} \geq 0, w_{n} \geq 0 $$

\subsubsection{Matrix Notation}

\begin{multicols}{2}

\underline{Primal}
$$ Max Z_{x} = CX $$

Subject to
$$ AX \leq b and X \geq 0 $$

\underline{Dual}
$$ Min Z_{w} = b^{T}W $$

Subject to
$$ A^{T}W \geq C^{T} and W \geq 0 $$

\end{multicols}

\subsection{Important characteristics of Duality}
\begin{enumerate}
    \item Dual of dual is primal.
    \item If either the primal or dual problem has a solution then the
        other also has a solution and their optimum values are equal.
    \item If any of the two problems has a infeasible solution, then the
        value of the objective function of the other is unbounded.
    \item The value of the objective function for any feasible solution
        of the primal is less than the value of the objective function
        for any feasible solution of the dual.
    \item If either the primal or dual has an unbounded solution, then
        the solution to the other problem is infeasible.
    \item If the primal has a feasilble solution, but the dual does not
        have then the primal will not have a finite optimum solution and
        vice versa.
    \item The final simplex table giving optimal solution of the primal
        also contains optimal solution of its dual in itself.
\end{enumerate}

\subsection{Advantages and Applications of Duality}
\begin{enumerate}
    \item Sometimes dual problem solution may be easier than primal
        solution, particularly when the number of decision variables is
        considerably less than slack / surplus variables.
    \item In the areas like economics, it is highly helpful in obtaining
        future decision in the activities being programmed.
    \item In physics, it is used in parallel circuit and series circuit
        theory.
    \item In game theory, dual is employed by column player who wishes
        to minimize his maximum loss while his opponent i.e. Row player
        applies primal to maximize his mimimum gains. However, if one
        problem is solved, the solution for other also can be obtained
        from the simplex tableau.
    \item When a problem does not yield any solution in primal, it can
        be verified with dual.
    \item Economic interpretations can be made and shadow prices can be
        determined enabling the managers to take further decisions.
\end{enumerate}

\subsection{Steps for a Standard Primal Form}
\begin{steps}
\item Change the objective function to Maximization form.
\item If the constraints have an inequality sign \lq$ \geq $\rq then
    multiply both sides by $ -1 $ and convert the inquality sign to
    \lq$ \leq $\rq.
\item If the constraints have an \lq$ = $\rq sign then replace it by two
    constraints involving the inequalities going in opposite directions.
    For example $ x_{1} + 2x_{2} = 4 $ is written as
    $$ x_{1} + 2x_{2} \leq 4 $$
    $ x_{1} + 2x_{2} \leq 4 $ (using step2) $ - x_{1} - 2x_{2} \leq -4 $
\item Every unrestricted variable is replaced by the difference of two
    non-negative variables.
\item We get the standard primal form of the given LPP in which.
    \begin{itemize}
        \item All constraints have \lq$ \leq $\rq sign, where the
            objective function is of maximization form.
        \item All constraints have \lq$ \geq $\rq sign, where the
            objective function is of minimization form.
    \end{itemize}
\end{steps}

\subsection{Rules for Converting any Primal into its Dual}
\begin{enumerate}
    \item Transpose the rows and columns of the constraint co-efficient.
    \item Transpose the co-efficient $ (c_{1}, c_{2}, \ldots c_{n}) $ of
        the objective function and the right side constants $ (b_{1},
        b_{2}, \ldots b_{n}) $.
    \item Change the inequalities from \lq$ \leq $\rq to \lq$ \geq $\rq 
        sign.
    \item Minimize the objective function instead of maximizing it.
\end{enumerate}

\subsection{Example Problems}
