\documentclass[a4paper]{article}
\usepackage[utf8]{inputenc}
\usepackage[margin=0.5in]{geometry}

\usepackage{graphicx}
\usepackage{tikz}
%\usetikzlibrary{positioning, backgrounds, calc, arrow.meta}
\usetikzlibrary{shapes.geometric, arrows}

% Set main font to Arial
\usepackage{helvet}
\renewcommand{\familydefault}{\sfdefault}

\begin{document}

\begin{center}
    \hrule
    \vspace{1em}
    {\LARGE \textbf{EMRS Teaching Aptitude}}
    \vspace{1em}
    \hrule
\end{center}

\section{Class 1}

\subsection{Teaching}
Nature, Characteristics, Objectives and Basic requirements, Learner's
characteristics, Factors affecting teaching.\\

Teaching is not merely imparting knowledge to students. Meaning of teaching can
be explained by narrow meaning and broad meaning. In the narrow sense, teaching
is a process which is imparted by a teacher or other person of the society to a
student at a particular place or a school.\\

In the broad sense, it is a process by which different stakeholders like the
family, the neighbours, friends, social and religious institutions, the
educational institutions etc. teach a person throughout his life as to how to
achieve his necessities and make an environment. Hence, according to wider
meaning, all the persons or things in the environment teach something or give
some experience.\\

Morrison (1934), Dewey (1934) expressed this concept of teaching by an
equation. \textbf{"Teaching is to learning as selling is to buying".}\\

Edmund Amidon defined teaching as "an interactive process, primarily involving
classroom talk which take place between teacher and pupil and occurs during
certain definable activities".\\

Davis et all., Gagne et al. have contributed significantly in defining this
concept and their views could be summarized as follows: Teaching is a
scientific process, and its major components are content, communication and
feedback.\\

Davis and Glaser (1962) have pointed out that the entire structure of teaching
has four steps -
\begin{itemize}
    \item \textbf{Step-I:} Planning of teaching which includes content
        analysis, identification and writing of objectives.
    \item \textbf{Step-II:} Organization of teaching which indicates the
        teaching strategies for achieving the objectives of teaching.
    \item \textbf{Step-III:} Identification of suitable teaching learning
        strategies for effective communication of content.
    \item \textbf{Step-IV:} Managing teaching-learning, whereby the focus is on
        the assessment of the learning objectives in terms of student 
        performance, and this forms the feedback to teacher and students.
\end{itemize}

B.O. Smith defined teaching as "Teaching is a system of actions intended to
induce learning".\\

Teacher require to things - 1. Knowledge, 2. Process of teaching.\\

Teaching aptitude encompasses a person's natural inclination, skills, and
knowledge necessary for effective teaching, assessing their ability to teach
and understand educational concepts. It involves understanding learner needs,
implementing effective teaching strategies across memory, understanding, and
reflective levels, and evaluating learning outcomes.\\

\subsection{Nature of Teaching}

\tikzstyle{sun} = [draw, ellipse]
\tikzstyle{planet} = [draw, rectangle]
\tikzstyle{arrow} = [thick,->,>=stealth]
\begin{figure}[h]
    \begin{center}
        \begin{tikzpicture}[node distance = 5em,auto]
            \node[sun] (a) {Nature of Teaching};
            \node[planet, above left of=a, xshift=-10em] (b) {
                Teaching is both Science and Art
            };
            \node[planet, above of=a, yshift=1em] (c) {Teaching is Complex};
            \node[planet, above right of=a, xshift=10em] (d) {
                Teaching can be Direct or Indirect
            };
            \node[planet, below left of=a, xshift=-10em] (e) {
                Teaching may be Planned or Unplanned
            };
            \node[planet, below of=a, yshift=-1em] (f) {
                Teaching Visualizes Change in Behavior
            };
            \node[planet, below right of=a, xshift=10em] (g) {
                Teaching can be Vertical or Horzontal
            };
            \draw [arrow] (a) -- (b);
            \draw [arrow] (a) -- (c);
            \draw [arrow] (a) -- (d);
            \draw [arrow] (a) -- (e);
            \draw [arrow] (a) -- (f);
            \draw [arrow] (a) -- (g);
        \end{tikzpicture}
        \caption{Nature of Teaching}
        \label{fig1}
    \end{center}
\end{figure}

\subsubsection{Teaching Visualizes Change in Behavior}
These changes can be in the cognitive (knowledge), psycho-motor (skills), and
affective (attitudes) values of the learners. The changes that take place in
the behaviour of learners as a result of learning should be tentatively
permanent.\\

\subsubsection{Teaching can be Vertical or Horizontal}
Depending on the objectives of teaching, teachers may lead students deep into
the topic. They not only help students know and understand the topic but teach
them higher order thinking skills like analysis, synthesis, evaluation and
creating. This type of teaching is known as vertical teaching. Conversely, if
the teachers teach one topic and then move on to more and more topics, they are
resorting to horizontal teaching. In that case, their teaching covers more
areas spreading over several topics instead of going deeper into one topic.\\

\subsubsection{Teaching as Art}
Elliot Eisner, in his book The Educational Imagination (1985) gives four
reasons to consider teaching as an art. They are as follows:
\begin{enumerate}
    \item Teaching can be performed with such skill and grace so that both
        teachers as well as learners experience the whole process
        aesthetically.
    \item Teacher's activities are dynamic i.e. they are influenced by various
        qualities and contingencies and change accordingly. They are not
        performed in strict routines and regimes.
    \item Teachers with good aesthetic sense usually incorporate them in their
        teaching process irrespective of the subject being taught.
    \item The end of teaching is not pre-decided, it is often created in the
        process.
\end{enumerate}
Thus the four senses of teaching are -
\begin{itemize}
    \item as a source of an aesthetic experience
    \item as dependent on the perception and control of qualities
    \item as a heuristic or adventitious activity
    \item as seeking emergent ends
\end{itemize}
these prove that teaching can be regarded as an art.

\subsubsection{Teaching as Science}

\section{Class 2}

\section{Class 3}

\section{Class 4}

\section{Class 5}

\section{Class 6}


\end{document}
